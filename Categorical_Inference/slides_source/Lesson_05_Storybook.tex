% print slides only
\documentclass[12pt,ignorenonframetext,aspectratio=169]{beamer}

% Print slides and notes
%\documentclass[12pt,ignorenonframetext,aspectratio=169,notes]{beamer}

% Print notes only
%\documentclass[12pt,ignorenonframetext,aspectratio=169,notes=only]{beamer}

% hack
\def\tightlist{}

\setbeamertemplate{navigation symbols}{}
\setbeamertemplate{caption}[numbered]
\setbeamertemplate{caption label separator}{:}
\setbeamercolor{caption name}{fg=normal text.fg}
\usepackage{amssymb,amsmath}
\usepackage{ifxetex,ifluatex}
\usepackage{graphicx}

\usepackage{fixltx2e} % provides \textsubscript
\usepackage{lmodern}

\usepackage{xcolor}
\definecolor{bryanBlue}{HTML}{306BA6}
\setbeamercolor{frametitle}{fg=bryanBlue,bg=white}
\setbeamertemplate{itemize items}[ball]
\setbeamercolor{item projected}{bg=bryanBlue}
\setbeamercolor{subitem projected}{bg=bryanBlue}
\setbeamercolor{title}{fg=bryanBlue,bg=white}

\ifxetex
  \usepackage{fontspec,xltxtra,xunicode}
  \defaultfontfeatures{Mapping=tex-text,Scale=MatchLowercase}
  \newcommand{\euro}{€}
\else
  \ifluatex
    \usepackage{fontspec}
    \defaultfontfeatures{Mapping=tex-text,Scale=MatchLowercase}
    \newcommand{\euro}{€}
  \else
    \usepackage[T1]{fontenc}
    \usepackage[utf8]{inputenc}
      \fi
\fi
% use upquote if available, for straight quotes in verbatim environments
\IfFileExists{upquote.sty}{\usepackage{upquote}}{}
% use microtype if available
\IfFileExists{microtype.sty}{\usepackage{microtype}}{}
\usepackage{color}
\usepackage{fancyvrb}
\newcommand{\VerbBar}{|}
\newcommand{\VERB}{\Verb[commandchars=\\\{\}]}
\DefineVerbatimEnvironment{Highlighting}{Verbatim}{commandchars=\\\{\}}
% Add ',fontsize=\small' for more characters per line
\usepackage{framed}
\definecolor{shadecolor}{RGB}{248,248,248}
\newenvironment{Shaded}{\begin{snugshade}}{\end{snugshade}}
\newcommand{\KeywordTok}[1]{\textcolor[rgb]{0.13,0.29,0.53}{\textbf{{#1}}}}
\newcommand{\DataTypeTok}[1]{\textcolor[rgb]{0.13,0.29,0.53}{{#1}}}
\newcommand{\DecValTok}[1]{\textcolor[rgb]{0.00,0.00,0.81}{{#1}}}
\newcommand{\BaseNTok}[1]{\textcolor[rgb]{0.00,0.00,0.81}{{#1}}}
\newcommand{\FloatTok}[1]{\textcolor[rgb]{0.00,0.00,0.81}{{#1}}}
\newcommand{\ConstantTok}[1]{\textcolor[rgb]{0.00,0.00,0.00}{{#1}}}
\newcommand{\CharTok}[1]{\textcolor[rgb]{0.31,0.60,0.02}{{#1}}}
\newcommand{\SpecialCharTok}[1]{\textcolor[rgb]{0.00,0.00,0.00}{{#1}}}
\newcommand{\StringTok}[1]{\textcolor[rgb]{0.31,0.60,0.02}{{#1}}}
\newcommand{\VerbatimStringTok}[1]{\textcolor[rgb]{0.31,0.60,0.02}{{#1}}}
\newcommand{\SpecialStringTok}[1]{\textcolor[rgb]{0.31,0.60,0.02}{{#1}}}
\newcommand{\ImportTok}[1]{{#1}}
\newcommand{\CommentTok}[1]{\textcolor[rgb]{0.56,0.35,0.01}{\textit{{#1}}}}
\newcommand{\DocumentationTok}[1]{\textcolor[rgb]{0.56,0.35,0.01}{\textbf{\textit{{#1}}}}}
\newcommand{\AnnotationTok}[1]{\textcolor[rgb]{0.56,0.35,0.01}{\textbf{\textit{{#1}}}}}
\newcommand{\CommentVarTok}[1]{\textcolor[rgb]{0.56,0.35,0.01}{\textbf{\textit{{#1}}}}}
\newcommand{\OtherTok}[1]{\textcolor[rgb]{0.56,0.35,0.01}{{#1}}}
\newcommand{\FunctionTok}[1]{\textcolor[rgb]{0.00,0.00,0.00}{{#1}}}
\newcommand{\VariableTok}[1]{\textcolor[rgb]{0.00,0.00,0.00}{{#1}}}
\newcommand{\ControlFlowTok}[1]{\textcolor[rgb]{0.13,0.29,0.53}{\textbf{{#1}}}}
\newcommand{\OperatorTok}[1]{\textcolor[rgb]{0.81,0.36,0.00}{\textbf{{#1}}}}
\newcommand{\BuiltInTok}[1]{{#1}}
\newcommand{\ExtensionTok}[1]{{#1}}
\newcommand{\PreprocessorTok}[1]{\textcolor[rgb]{0.56,0.35,0.01}{\textit{{#1}}}}
\newcommand{\AttributeTok}[1]{\textcolor[rgb]{0.77,0.63,0.00}{{#1}}}
\newcommand{\RegionMarkerTok}[1]{{#1}}
\newcommand{\InformationTok}[1]{\textcolor[rgb]{0.56,0.35,0.01}{\textbf{\textit{{#1}}}}}
\newcommand{\WarningTok}[1]{\textcolor[rgb]{0.56,0.35,0.01}{\textbf{\textit{{#1}}}}}
\newcommand{\AlertTok}[1]{\textcolor[rgb]{0.94,0.16,0.16}{{#1}}}
\newcommand{\ErrorTok}[1]{\textcolor[rgb]{0.64,0.00,0.00}{\textbf{{#1}}}}
\newcommand{\NormalTok}[1]{{#1}}
\usepackage{graphicx}
\makeatletter
\def\maxwidth{\ifdim\Gin@nat@width>\linewidth\linewidth\else\Gin@nat@width\fi}
\def\maxheight{\ifdim\Gin@nat@height>\textheight0.8\textheight\else\Gin@nat@height\fi}
\makeatother
% Scale images if necessary, so that they will not overflow the page
% margins by default, and it is still possible to overwrite the defaults
% using explicit options in \includegraphics[width, height, ...]{}
\setkeys{Gin}{width=\maxwidth,height=\maxheight,keepaspectratio}

% Comment these out if you don't want a slide with just the
% part/section/subsection/subsubsection title:
\AtBeginPart{
  \let\insertpartnumber\relax
  \let\partname\relax
  \frame{\partpage}
}
\AtBeginSection{
  \let\insertsectionnumber\relax
  \let\sectionname\relax
  \frame{\sectionpage}
}
\AtBeginSubsection{
  \let\insertsubsectionnumber\relax
  \let\subsectionname\relax
  \frame{\subsectionpage}
}

\setlength{\parindent}{0pt}
\setlength{\parskip}{6pt plus 2pt minus 1pt}
\setlength{\emergencystretch}{3em}  % prevent overfull lines
\setcounter{secnumdepth}{0}

\title{Inference for Categorical Data}
\date{}

\newcommand{\columnsbegin}{\begin{columns}}
\newcommand{\columnsend}{\end{columns}}

\begin{document}
\frame{\titlepage}

\begin{frame}{Categorical Variables}

\begin{itemize}
\item
  Non-numerical, non-overlapping categories
\item
  Frequencies or Counts
\item
  Proportions
\item
  Frequency Distribution Tables
\item
  Contingency Tables
\end{itemize}

\note{A categorical variable is a variable which takes on values from
non-numerical, non-overlapping categories. These are also called
qualitative variables.

Rather than finding means and standard deviations, we tally up the
number of observations in a sample or population that fall within each
category. These are called frequencies or counts. From these we can
compute relative frequencies which we also call proportions and we can
also find percentages.

When summarizing just one categorical variable, the counts are placed in
a frequency distribution table. The frequencies for the
cross-classification of two categorical variables are placed in a
contingency table.}

\end{frame}

\begin{frame}{Fast Facts: One-Sample \emph{Z} Procedures for a
Proportion}

\begin{table}
        \centering
        \begin{tabular}{ll}
        \bf{Why}: & Hypothesis test - To $compare$ an unknown population proportion to \\
                  & some hypothetical value. \\ 
                  & Confidence Interval - To $estimate$ an unknown population proportion. \\
                  &  \\
        \bf{When}: & The following conditions are necessary for these procedures to be \\
                  & accurate and valid.   \\
                  & \hspace{1em} 1. The sample is selected randomly \\
                  & \hspace{1em} 2. The sample contains at least 10 successes and 10 failures  \\

                  &  \\
        \bf{How}: & Use R function \bf{prop.test()}  \\
        \end{tabular}
\end{table}

\note{No audio}

\end{frame}

\begin{frame}{Review of Inference for Proportions - CI for a Single
Population Proportion}

The following R code reproduces the computations for the confidence
interval in Example 10.5 on pp.~506-507 of the Ott textbook

\begin{center}
prop.test(1200,2500,p=.44,correct=FALSE)
\end{center}

\note{You may want to grab your textbook to follow along with the next
few slides as we review hypothesis tests and confidence intervals for
one and two population proportions.

The R function prop.test is used both cases.

The option correct=FALSE is turning off the Yates continuity correction,
which can overcompensate with larger sample sizes. The default in R is
to apply the Yates continuity correction in prop.test.}

\end{frame}

\begin{frame}[fragile]{R output for the confidence interval in Example
10.5, pp.~506-507}

\begin{verbatim}
## 
##  1-sample proportions test without continuity correction
## 
## data:  1200 out of 2500
## X-squared = 16.234, df = 1, p-value = 5.599e-05
## alternative hypothesis: true p is not equal to 0.44
## 95 percent confidence interval:
##  0.4604617 0.4995996
## sample estimates:
##    p 
## 0.48
\end{verbatim}

\note{Looking on page 507 of Ott's textbook, we can see that the
confidence interval produce by R with a lower bound of 0.46 and an upper
bound of .499, which would round to .50, matches exactly the confidence
interval for a single population proportion in the textbook example.

bottom panel note: the R function binom.test does the same, only
provides the interval or test based on the exact binomial distribution
rather than the normal approximation}

\end{frame}

\begin{frame}{Review of Inference for Proportions - HT for a Single
Population Proportion}

The following R code reproduces the computations for the hypothesis test
in Example 10.5 on pp.~506-507 of the Ott textbook

\begin{center}
prop.test(1200,2500,p=.44,alternative="greater",correct=FALSE)
\end{center}

\note{Since the hypothesis test of Example 10.5 is one-sided, with the
alternative hypothesis of the population proportion pi being greater
than .44, we specify the alternative greater in R.

Note that we can simply enter the number of successes, 1200, and the
sample size, 2500, directly into the prop.test function.

bottom panel note: Enter ?prop.test in R to see more}

\end{frame}

\begin{frame}[fragile]{R output for the hypothesis test in Example 10.5}

\begin{verbatim}
## 
##  1-sample proportions test without continuity correction
## 
## data:  1200 out of 2500
## X-squared = 16.234, df = 1, p-value = 2.799e-05
## alternative hypothesis: true p is greater than 0.44
## 95 percent confidence interval:
##  0.4635951 1.0000000
## sample estimates:
##    p 
## 0.48
\end{verbatim}

\note{Here is the R output for the one-sample test for a population
proportion without the Yates' continuity correction. Chi-square with 1
df is z squared (here R reports 16.234, the square root of which is 4.03
- with the difference from the textbook's z of 4.00 due to rounding).
The textbook states the p-value as .00003 - here we see the p-value in
scientific notation as 2.799 times 10 to the negative 5th - which when
rounded, is .00003}

\end{frame}

\begin{frame}{Fast Facts: Two-Sample \emph{Z} Procedures for
Proportions}

\begin{table}
        \centering
        \begin{tabular}{ll}
        \bf{Why}: & Hypothesis test - To $compare$ two unknown population proportions. \\
                  & Confidence Interval - To $estimate$ the difference between two unknown \\
                  & population proportions. \\
                  &  \\
        \bf{When}: & The following conditions are necessary for these procedures to be \\
                  & accurate and valid.   \\
                  & \hspace{1em} 1. The sample is selected randomly \\
                  & \hspace{1em} 2. The samples are selected independently  \\
                  & \hspace{1em} 3. Both samples contains at least 10 successes and 10 failures  \\

                  &  \\
        \bf{How}: & Use R function \bf{prop.test()}  \\
        \end{tabular}
\end{table}

\note{No audio}

\end{frame}

\begin{frame}{Review of Inference for Proportions - CI for a Difference
in Population Proportions}

The following R code reproduces the computations for the confidence
interval in Example 10.6 on pp.~508-509 of the Ott textbook

\begin{table}
\centering
\begin{tabular}{ll}
\hspace{3em} & aware=c(413,392) \\
\hspace{3em} & interviewed=c(527,608) \\
\hspace{3em} & prop.test(aware,interviewed,correct=FALSE) \\
\end{tabular}
\end{table}

\note{One way to enter data for either a confidence interval or a
hypothesis test concerning a difference in population proportions in
prop.test is as a vector of the number of successes and a vector of the
corresponding sample sizes. Here, for Example 10.6 from Table 10.1 on
page 509 in the textbook, the number in the sample who are aware of the
product are in the vector called ``aware'' and the sample sizes are in
the vector called ``interviewed.''}

\end{frame}

\begin{frame}{Table 10.1 (for Example 10.6, p.~509 in Ott)}

\begin{figure}[htbp]
\centering
\includegraphics{./figures/Table_10-1.jpg}
\caption{}
\end{figure}

\note{}

\end{frame}

\begin{frame}[fragile]{R output for the confidence in Example 10.6}

\begin{verbatim}
## 
##  2-sample test for equality of proportions without continuity
##  correction
## 
## data:  aware out of interviewed
## X-squared = 26.429, df = 1, p-value = 2.734e-07
## alternative hypothesis: two.sided
## 95 percent confidence interval:
##  0.08714759 0.19074115
## sample estimates:
##    prop 1    prop 2 
## 0.7836812 0.6447368
\end{verbatim}

\note{}

\end{frame}

\begin{frame}{Review of Inference for Proportions - HT for a Difference
in Population Proportions}

The following R code reproduces the computations for the hypothesis test
in Example 10.7 on pp.~510-511 of the Ott textbook

\begin{table}
\centering
\begin{tabular}{ll}
\hspace{3em} & exam=matrix(c(94,113,31,62),nrow=2) \\
\hspace{3em} & stats::prop.test(exam,correct=FALSE,alternative='greater') \\
\end{tabular}
\end{table}

\note{In this example I wanted to show you a different way that R can
take the data. It can be entered as a 2x2 matrix with the two columns
giving counts of successes and failures, respectively. The successes go
in column 1, the failures go in column 2.

It was necessary to specify that we wanted the stats package here
because another package that has been installed for this lesson called
mosaic also has a function called prop.test - which behaves a little
differently, so here we must specify which package we want to call the
prop.test from, which is the package called stats, so the prop.test
function is preceded by stats followed by two colons.

bottom panel note: Counts are from Table 10.2 on p.~510.}

\end{frame}

\begin{frame}[fragile]{R output for the contingency table for Example
10.7}

\begin{figure}[htbp]
\centering
\includegraphics{./figures/Table_10-2.jpg}
\caption{}
\end{figure}

\begin{Shaded}
\begin{Highlighting}[]
\NormalTok{exam=}\KeywordTok{matrix}\NormalTok{(}\KeywordTok{c}\NormalTok{(}\DecValTok{94}\NormalTok{,}\DecValTok{113}\NormalTok{,}\DecValTok{31}\NormalTok{,}\DecValTok{62}\NormalTok{),}\DataTypeTok{nrow=}\DecValTok{2}\NormalTok{)}
\NormalTok{exam}
\end{Highlighting}
\end{Shaded}

\begin{verbatim}
##      [,1] [,2]
## [1,]   94   31
## [2,]  113   62
\end{verbatim}

\note{Note the matrix is entered in R so that the counts of successes
are in column 1 - these are the ones who passed the exam in Example
10.7, see Table 10.2 on page 510 - and the counts for failures are in
the second column - these are the ones who didn't pass the English
language exam in Example 10.7.

bottom panel note: See Table 10.2 on page 510 of the Ott textbook}

\end{frame}

\begin{frame}[fragile]{R output for the hypothesis test in Example 10.7}

\begin{verbatim}
## 
##  2-sample test for equality of proportions without continuity
##  correction
## 
## data:  exam
## X-squared = 3.8509, df = 1, p-value = 0.02486
## alternative hypothesis: greater
## 95 percent confidence interval:
##  0.01926052 1.00000000
## sample estimates:
##    prop 1    prop 2 
## 0.7520000 0.6457143
\end{verbatim}

\note{}

\end{frame}

\begin{frame}{Fisher Exact Test}

The following R code reproduces the computations for the hypothesis test
in Example 10.8 on pp.~512-513 of the Ott textbook.

\begin{table}
\centering
\begin{tabular}{ll}
\hspace{3em} & count=matrix(c(38,14,4,7),nrow=2) \\
\hspace{3em} & fisher.test(count,alternative="greater") \\
\end{tabular}
\end{table}

\note{Bottom panel note: Fisher's Exact test is used when at least one
of the expected cell counts in a 2x2 table is under 5.}

\end{frame}

\begin{frame}{Table 10.4 for Example 10.8, p.~512 in Ott text}

\begin{figure}[htbp]
\centering
\includegraphics{./figures/Table_10-4.jpg}
\caption{}
\end{figure}

\note{}

\end{frame}

\begin{frame}[fragile]{R setup of the contingency table for Example
10.8}

\begin{Shaded}
\begin{Highlighting}[]
\NormalTok{count=}\KeywordTok{matrix}\NormalTok{(}\KeywordTok{c}\NormalTok{(}\DecValTok{38}\NormalTok{,}\DecValTok{14}\NormalTok{,}\DecValTok{4}\NormalTok{,}\DecValTok{7}\NormalTok{),}\DataTypeTok{nrow=}\DecValTok{2}\NormalTok{)}
\NormalTok{count}
\end{Highlighting}
\end{Shaded}

\begin{verbatim}
##      [,1] [,2]
## [1,]   38    4
## [2,]   14    7
\end{verbatim}

\[H_0 \text{: } \pi_P \ge \pi_{PV}\] \[H_a \text{: } \pi_P < \pi_{PV}\]

\note{Notice that we only need to enter the inner cells of the 2x2 table
- not the row and column totals in the margins of the table. R will
compute them internally and use them as needed to compute the p-value
for the Fisher Exact Test.

If you're looking at the hyotheses on the bottom of page 512, You'll
notice that the alternative says the proportion for drug P (indicated by
pi\_P) is LESS than the proportion for drug PV, but recall that in the R
code we had specified the alternative ``greater.'' This is because the
drug PV outcomes are listed in the first row of the 2x2 table. Be
careful with one-sided tests to code them in the right direction.

"}

\end{frame}

\begin{frame}[fragile]{R output for the Fisher Exact Test in Example
10.8}

\begin{verbatim}
## 
##  Fisher's Exact Test for Count Data
## 
## data:  count
## p-value = 0.02537
## alternative hypothesis: true odds ratio is greater than 1
## 95 percent confidence interval:
##  1.22629     Inf
## sample estimates:
## odds ratio 
##   4.615064
\end{verbatim}

\note{As the textbook states, Fisher's Exact Test computes the p-value
as the sum of the probabilities for all tables having 38 or more
successes for the drug PV.

Also, testing the that proportion of successes for PV is greater than
for drug P is equivalent to saying the odds ratio is greater than 1.
We'll get to odds ratios a bit later in these slides.}

\end{frame}

\begin{frame}{Chi-Square Tests}

\begin{itemize}
\tightlist
\item
  One categorical variable

  \begin{itemize}
  \tightlist
  \item
    Goodness-of-fit test
  \end{itemize}
\item
  Two categorical variables

  \begin{itemize}
  \tightlist
  \item
    Test for independence
  \item
    Test for homogeneity
  \end{itemize}
\end{itemize}

\note{When just one categorical variable is under consideration, the
chi-square test for goodness-of-fit can be used to test the hypothesis
that the sample was drawn from a specified distribution vs the
alternative that is was not. You may recall that the Shapiro-Wilk test
for normality is also a goodness-of-fit test.

For two categorical factors, the chi-square statistic can be used to
test for the independence of the two factors vs the alternative that the
factors are associated. With the test for independence, the sampling
scheme must be that a random sample has been drawn from the population
of interest, thus making the row and column totals random counts.

The chi-square test for homogeneity has identical computations for the
test statistic and p-value are, but differs in the sampling scheme -
which also affects how the hypotheses are set up and how to interpret
the results. For the test of homogeneity, independent samples are drawn
from the subpopulations defined by the categories of one factor, in
order to compare the distributions of the other categorical variable.
The null hypothesis would be that all of the distributions are the same
against the alternative that they are not all the same.}

\end{frame}

\begin{frame}{Fast Facts: Chi-Square Test for Goodness-of-Fit}

\begin{table}
        \centering
        \begin{tabular}{ll}
        \bf{Why}: & To determine whether or not a sample was drawn from a particular  \\ 
                  & distribution with hypothesized proportions for specified categories. \\
                  &  \\
        \bf{When}: & The following conditions are necessary for this procedure to be \\
                  & accurate and valid.  \\
                  & \hspace{1em} 1. The samples are selected randomly \\
                  & \hspace{1em} 2. The sample is large enough so that the expected cell   \\
                  & \hspace{2.3em} frequencies are all at least 5 \\
                  &  \\
        \bf{How}: & Use R function \bf{chisq.test()}  \\
        \end{tabular}
\end{table}

\note{No audio}

\end{frame}

\begin{frame}{Example A: Chi-square GOF test}

Suppose it is reported in a media release that 24\% of all personal
loans are for home mortgages, 38\% were for automobile purchases, 18\%
were for credit card loans, and the rest were for other types of loans.
Records for a random sample of 55 loans was obtained and each was
classified into one of these categories. The results are in the
following table.

\begin{table}
\centering
\begin{tabular}{r|c|c|c|c}
 & Mortgage & Auto & Credit & Other  \\
\hline
Number of loans & 24 & 21 & 6 & 4 \\
\end{tabular}
\end{table}

\note{}

\end{frame}

\begin{frame}{GOF Test: The Request}

Conduct the appropriate test to determine if the distribution reported
in the media release for the frequency of the types of loans fits the
actual distribution of types loans in the population. Use
\(\alpha = 0.01\).

\note{}

\end{frame}

\begin{frame}{GOF Test: The Hypotheses}

\(H_0 \text{: } \pi_{Mortgage}=0.24, \pi_{Auto}=0.38, \pi_{Credit}=0.18, \pi_{Other}=0.20\)

\(H_a \text{: At least one } \pi_i \text{ differs from its hypothesized value}\)

\note{Verbally, the null hypothesis is claiming that the distribution
claimed by the media release is correct. The alternative hypothesis is
simply that the distribution is not correct since at least one of the
hypothesized probabilities is not right.

Many times, the chi square goodness-of-fit test is used to determine if
the categories have equal probabilities - like testing to see if a die
is fair, for example. In those cases it isn't necessary to specify the
proportions because they are self evident. If a 6-sided die is equally
balanced, then each outcome should have a probability of 1 out of 6. If
we were testing to see if the proportions of loans were equally likely
here, the null hypothesis probabilities would all be one fourth, since
there are 4 categories.}

\end{frame}

\begin{frame}[fragile]{GOF Test: Getting the Data into R}

\begin{Shaded}
\begin{Highlighting}[]
\NormalTok{observed=}\KeywordTok{c}\NormalTok{(}\DecValTok{24}\NormalTok{,}\DecValTok{21}\NormalTok{,}\DecValTok{6}\NormalTok{,}\DecValTok{4}\NormalTok{)}
\NormalTok{proportions=}\KeywordTok{c}\NormalTok{(.}\DecValTok{24}\NormalTok{,.}\DecValTok{38}\NormalTok{,.}\DecValTok{18}\NormalTok{,.}\DecValTok{20}\NormalTok{)}
\end{Highlighting}
\end{Shaded}

\note{We simply create a vector that contains the observed cell counts,
here I named it ``observed'' and a vector holding the hypothesized
proportions, which I called ``proportions.''

You have probably noticed by now that we are using the terms proportions
and probabilties interchangeably.

In this test our presumption is that the underlying variable has a
multinomial probability distribution with the probabilities specified in
the null hypothesis. Multinomial distributions are characterized by
having n identical, independent trials, each having k possible outcomes,
where the probabilities of each of the k outcomes remains constant from
trial to trial.}

\end{frame}

\begin{frame}[fragile]{GOF Test: Getting the Test Statistic \&
\(P\)-value in R}

\begin{Shaded}
\begin{Highlighting}[]
\KeywordTok{chisq.test}\NormalTok{(}\DataTypeTok{x=}\NormalTok{observed,}\DataTypeTok{p=}\NormalTok{proportions)}
\end{Highlighting}
\end{Shaded}

\begin{verbatim}
## 
##  Chi-squared test for given probabilities
## 
## data:  observed
## X-squared = 14.828, df = 3, p-value = 0.00197
\end{verbatim}

\note{One quick check to see that we have coded it right is to look at
the degrees of freedom. It should be the number of categories minus 1.
Since there were 4 loan categories being tested and we see the degrees
of freedom given as 3, we should start to get warm fuzzies about now.

What should we conclude? Was the media report correct? No, according to
the sample data resulting in a test statistic of 14.828 and a p-value of
.00197, which is less than .01, we should reject the null hypothesis and
claim that the actual distribution for the types of personal loans is
different from what was reported.}

\end{frame}

\begin{frame}[fragile]{GOF Test: Checking Expected Values in R}

\begin{Shaded}
\begin{Highlighting}[]
\DecValTok{55}\NormalTok{*proportions}
\end{Highlighting}
\end{Shaded}

\begin{verbatim}
## [1] 13.2 20.9  9.9 11.0
\end{verbatim}

\note{With a smaller sample like this one, it would behoove us to check
the sample size requirement for this chi-square test. You see the
expected cell frequencies are easily obtained by multiplying the vector
of hypothesized proportions by the sample size.

Notice also that the requirement isn't that the observed counts are all
at least 5, but that the expected counts are all at least 5. So even
though there was an observed cell frequency of 4 here, our sample was
still large enough to trust the chi-square test for goodness-of-fit
here, at least we can trust it to the extent that we didn't just make a
Type 1 error - which was controlled at the 1\% level of significance in
this test.}

\end{frame}

\begin{frame}[fragile]{Making a Bar Graph for One Categorical Variable}

\vspace{-.7em}

\begin{Shaded}
\begin{Highlighting}[]
\NormalTok{observed=}\KeywordTok{c}\NormalTok{(}\DecValTok{24}\NormalTok{,}\DecValTok{21}\NormalTok{,}\DecValTok{6}\NormalTok{,}\DecValTok{4}\NormalTok{)}
\KeywordTok{barplot}\NormalTok{(observed,}\DataTypeTok{names.arg=}\KeywordTok{c}\NormalTok{(}\StringTok{"Mort"}\NormalTok{,}\StringTok{"Auto"}\NormalTok{,}\StringTok{"Credit"}\NormalTok{,}\StringTok{"Other"}\NormalTok{),}
        \DataTypeTok{ylab=}\StringTok{"Count"}\NormalTok{,}\DataTypeTok{col=}\KeywordTok{c}\NormalTok{(}\StringTok{"purple"}\NormalTok{,}\StringTok{"blue"}\NormalTok{,}\StringTok{"red"}\NormalTok{,}\StringTok{"green"}\NormalTok{))}
\end{Highlighting}
\end{Shaded}

\vspace{-1.3em}

\begin{figure}[htbp]
\centering
\includegraphics{./figures/Bar1.png}
\caption{}
\end{figure}

\note{No audio.

Bottom Panel Note:

See the .Rmd file for the code.}

\end{frame}

\begin{frame}[fragile]{What About Data in a Larger Data File?}

The HealthExam data frame contains a variety of both quantitative and
categorical variables for 80 patients. In the file, the variable Region
indicates the region of the U.S. for each of the 80 patients.

Consider the following table of counts for patients falling in the four
regions of the U.S.

\begin{Shaded}
\begin{Highlighting}[]
\KeywordTok{data}\NormalTok{(}\StringTok{"HealthExam"}\NormalTok{)}
\KeywordTok{table}\NormalTok{(HealthExam$Region)}
\end{Highlighting}
\end{Shaded}

\begin{verbatim}
## 
##   Midwest Northeast     South      West 
##        16        22        20        22
\end{verbatim}

\note{No audio.

Bottom panel note:\\
You should load the HealthExam data file into your R session and take a
look at it. It is in the DS705Data package.}

\end{frame}

\begin{frame}[fragile]{GOF Test on Variable From Larger Data File}

Test that the regions are equally represented in the population. Use
\(\alpha=0.05\).

\begin{quote}
$H_0 \text{: } \pi_{Midwest}=0.25, \pi_{Northeast}=0.25, \pi_{South}=0.25, \pi_{West}=0.25$

$H_a \text{: At least one } \pi_i \text{ differs from its hypothesized value}$
\end{quote}

\begin{Shaded}
\begin{Highlighting}[]
\NormalTok{observed=}\KeywordTok{table}\NormalTok{(HealthExam$Region)  }\CommentTok{# get observed cell counts }
\NormalTok{proportions=}\KeywordTok{c}\NormalTok{(.}\DecValTok{25}\NormalTok{,.}\DecValTok{25}\NormalTok{,.}\DecValTok{25}\NormalTok{,.}\DecValTok{25}\NormalTok{)  }\CommentTok{# specify proportions from H0}
\KeywordTok{chisq.test}\NormalTok{(}\DataTypeTok{x=}\NormalTok{observed,}\DataTypeTok{p=}\NormalTok{proportions) }\CommentTok{# test for goodness-of-fit}
\end{Highlighting}
\end{Shaded}

\begin{verbatim}
## 
##  Chi-squared test for given probabilities
## 
## data:  observed
## X-squared = 1.2, df = 3, p-value = 0.753
\end{verbatim}

\note{If you are pulling counts for a categorical variable out of a
larger data frame with the table function, you can easily store the
counts in an object and place them into the chisq.test function exactly
as if you specified them yourself. You will, however, have to provide
hypothesized proportions for the goodness-of-fit test. In this case, we
are testing to see of the proportions are all equal to each other, so
since there are 4 categories, each hypothetical proportion will be 0.25.

With a p-value of 0.753, H0 will not be rejected at a 5\% level of
significance. There is not enough evidence to conclude that any of the
proportions for the geographic regions differ from 0.25 in the
population that this sample represents.}

\end{frame}

\begin{frame}{Fast Facts: Chi-Square Test for Independence}

\begin{table}
        \centering
        \begin{tabular}{ll}
        \bf{Why}: & To determine if two categorical variables (called factors) are   \\ 
                  &  associated or independent.\\
                  &  \\
        \bf{When}: & The following conditions are necessary for this procedure to be \\
                  & accurate and valid.  \\
                  & \hspace{1em} 1. The samples are selected randomly \\
                  & \hspace{1em} 2. The sample is large enough so that the expected cell   \\
                  & \hspace{2.3em} frequencies are all at least 5 \\
                  &  \\
        \bf{How}: & Use R function \bf{chisq.test()}  \\
        \end{tabular}
\end{table}

\note{No audio}

\end{frame}

\begin{frame}[fragile]{Example B: Health Exam Data}

The Age Group and Region for the first 6 out of 80 subjects is as
follows

\begin{verbatim}
##   HealthExam.AgeGroup HealthExam.Region
## 1            36 to 64              West
## 2            36 to 64             South
## 3                 65+           Midwest
## 4            36 to 64              West
## 5            36 to 64         Northeast
## 6                 65+           Midwest
\end{verbatim}

\note{Instead of having the counts as basic summary statistics for our
categorical variables, we may have a large data frame that contains the
individual observations. That's OK. R will know just what to do with
them and they can be entered into the chisq.test function in the same
way as the vectors or matrices containing the frequencies.}

\end{frame}

\begin{frame}[fragile]{Example B: Health Exam Data Contingency Table}

To see the crosstabs, use the `table' function in R

\begin{Shaded}
\begin{Highlighting}[]
\NormalTok{tbl <-}\StringTok{ }\KeywordTok{with}\NormalTok{(HealthExam,}\KeywordTok{table}\NormalTok{(AgeGroup,Region))}
\KeywordTok{addmargins}\NormalTok{(tbl)}
\end{Highlighting}
\end{Shaded}

\begin{verbatim}
##           Region
## AgeGroup   Midwest Northeast South West Sum
##   18 to 35       6         9     5    8  28
##   36 to 64       4         7    13    8  32
##   65+            6         6     2    6  20
##   Sum           16        22    20   22  80
\end{verbatim}

\note{When your data comes as individual observations in a data frame,
it is a good idea to just look at the counts to get a feel for what
relationship might exist between the factors and to make sure that there
aren't any unexpected surprises in your data set.

NEW AUDIO: Note that the addmargins function will display the row and
column totals.}

\end{frame}

\begin{frame}[fragile]{Making a Bar Graph for Two Categorical Variables}

\vspace{-.7em}

\begin{Shaded}
\begin{Highlighting}[]
\KeywordTok{barplot}\NormalTok{(tbl,}\DataTypeTok{xlab=}\StringTok{"Region"}\NormalTok{, }\DataTypeTok{ylab=}\StringTok{"Frequency"}\NormalTok{, }
        \DataTypeTok{col=}\KeywordTok{c}\NormalTok{(}\StringTok{"khaki"}\NormalTok{,}\StringTok{"cyan"}\NormalTok{,}\StringTok{"coral"}\NormalTok{),}\DataTypeTok{legend=}\KeywordTok{rownames}\NormalTok{(tbl),}\DataTypeTok{beside=}\NormalTok{T) }
\end{Highlighting}
\end{Shaded}

\vspace{-1.3em}

\begin{figure}[htbp]
\centering
\includegraphics{./figures/Bar2.png}
\caption{}
\end{figure}

\note{No audio.

Bottom panel note:

R has many fun colors to choose from! Search the web for the names.}

\end{frame}

\begin{frame}{R Code for Tests of Independence or Homogeneity}

Whether it is a test for independence or homogeneity, the R code is the
same.

chisq.test(AgeGroup,Region,data=HealthExam)

\note{The chisq.test function can be used with vectors or matrices
containing the contingency table frequencies in the same way that was
shown for the prop.test function previously in this presentation.

However, when categorical data is listed out in a data frame, the
variables can be loaded directly into the chisq.test function by their
names in the data frame.

NEW Bottom Panel Note:

A table can be stored and used directly in the chisq.test() function as
chisq.test(tbl).}

\end{frame}

\begin{frame}[fragile]{Example B: Health Exam Output from chisq.test}

Since the 80 people selected in this study randomly fell into the age
categories and geographic regions, the chi-square test here is for
independence (not homogeneity).

\begin{verbatim}
## 
##  Pearson's Chi-squared test
## 
## data:  HealthExam$AgeGroup and HealthExam$Region
## X-squared = 8.188, df = 6, p-value = 0.2247
\end{verbatim}

\note{}

\end{frame}

\begin{frame}{Chi-square test for Health Exam data}

\begin{table}
\centering
\begin{tabular}{ll}
\hspace{3em} & $H_0 \text{: Age Group and Region are independent.}$ \\
\hspace{3em} & $H_a \text{: Age Group and Region are associated.}$ \\
\end{tabular}
\end{table}

Conclusion: Do not reject \(H_0\) at \(\alpha=0.05\). There is
insufficient evidence in this sample to claim that Age Group and Region
are associated for the population of U.S. adults (\(P=0.2247\)).

\note{You see the conclusion here is to not reject the null hypothesis .
. .But wait! some of those cell counts were pretty small - we should
check the expected cell counts to see if any are under 5.}

\end{frame}

\begin{frame}[fragile]{Expected Cell Counts for Health Exam data}

\begin{Shaded}
\begin{Highlighting}[]
\NormalTok{result=}\KeywordTok{chisq.test}\NormalTok{(HealthExam$AgeGroup,HealthExam$Region)}
\NormalTok{result$expected}
\end{Highlighting}
\end{Shaded}

\begin{verbatim}
##                    HealthExam$Region
## HealthExam$AgeGroup Midwest Northeast South West
##            18 to 35     5.6       7.7     7  7.7
##            36 to 64     6.4       8.8     8  8.8
##            65+          4.0       5.5     5  5.5
\end{verbatim}

\note{To get the expected cell counts you see that its necessary to
assign the chisq.test output to an object in R and then call from that
object the expected values using this code here ``result dollar sign
expected.''

Do you see that the expected cell frequency for the 65 and over age
group in the Midwest REgion is 4? While it is only one cell count, and
it is very close to 5, even so, using the chi-square distribution for
the test statistic may not be such a good approximation, even to the
extent that we should at least look at another test - one that can
handle small expected cell frequencies. Fisher's Exact Test is just the
one. It can handle tables larger than 2x2. Let's see what is says about
the Health Exam data.}

\end{frame}

\begin{frame}[fragile]{Fishers Exact Test for Health Exam data - more
than a 2x2 table}

\begin{Shaded}
\begin{Highlighting}[]
\KeywordTok{fisher.test}\NormalTok{(HealthExam$AgeGroup,HealthExam$Region)}
\end{Highlighting}
\end{Shaded}

\begin{verbatim}
## 
##  Fisher's Exact Test for Count Data
## 
## data:  HealthExam$AgeGroup and HealthExam$Region
## p-value = 0.2443
## alternative hypothesis: two.sided
\end{verbatim}

\note{Bottom panel note: Note that in this case the result is nearly
identical to the chi-square test.}

\end{frame}

\begin{frame}[fragile]{Row Percents}

\begin{Shaded}
\begin{Highlighting}[]
\KeywordTok{options}\NormalTok{(}\DataTypeTok{digits=}\DecValTok{3}\NormalTok{)}
\NormalTok{demographics=}\KeywordTok{table}\NormalTok{(HealthExam$AgeGroup,HealthExam$Region)}
\KeywordTok{prop.table}\NormalTok{(demographics,}\DecValTok{1}\NormalTok{)*}\DecValTok{100}
\end{Highlighting}
\end{Shaded}

\begin{verbatim}
##           
##            Midwest Northeast South West
##   18 to 35    21.4      32.1  17.9 28.6
##   36 to 64    12.5      21.9  40.6 25.0
##   65+         30.0      30.0  10.0 30.0
\end{verbatim}

\note{If I was interested in looking at the distribution of people in
the 4 geographic Regions for each Age Group. Base on the way the
contingency table is arranged, I would need row percents. That is, the
rows add up to 100 percent.

Comparisons of percentages among Age Groups can now be made for each
Region. So I can say something like ``21.4\% of the all people in the
sample age 18 to 35 live in the Midwest, while only 12.5\% of the 36 to
65 year-olds live in the Midwest and 30\% of people over 65 live in the
Midwest.''

These percentages may seem far apart, but they weren't different enough
for our chi-square test here to reject the hypothesis of independence.
The sample size is big enough to conduct the chi-square test, but it is
still a relatively small sample and so it has a lower power to detect
differences.

prop.table will give them as proportions and multiplying by 100 converts
them to percents, which may be easier for us to think about.

The number 1 in the prop.table function is what directs R to compute row
percents. Just try to remember that in matrix notation, the rows get
mentioned first, so a 1 is appropriate.

Setting options to 3 is a global setting which will print numerical
values to 3 digits. The default in R is 7 digits. \# makes the table
easier to read.

P.S. Don't get too excited about the actual numbers in this table - they
are more for the academic exercise than for their accuracy in describing
current US demographics. In reality I'm guessing that more than 10
percent of people over 65 live in the South, but I digress . . .}

\end{frame}

\begin{frame}[fragile]{Column Percents}

\begin{Shaded}
\begin{Highlighting}[]
\KeywordTok{options}\NormalTok{(}\DataTypeTok{digits=}\DecValTok{3}\NormalTok{)}
\NormalTok{demographics=}\KeywordTok{table}\NormalTok{(HealthExam$AgeGroup,HealthExam$Region)}
\KeywordTok{prop.table}\NormalTok{(demographics,}\DecValTok{2}\NormalTok{)*}\DecValTok{100}
\end{Highlighting}
\end{Shaded}

\begin{verbatim}
##           
##            Midwest Northeast South West
##   18 to 35    37.5      40.9  25.0 36.4
##   36 to 64    25.0      31.8  65.0 36.4
##   65+         37.5      27.3  10.0 27.3
\end{verbatim}

\note{If I was interested in looking at the distribution of people in
the 3 Age Groups for each Region. Base on the way the contingency table
is arranged, I would need row percents. Notice for this one it is the
columns that add up to 100 percent.

Comparisons of percentages among Geographic Regions can now be made for
each Age Group. So I can say something like ``37.5\% of the all people
in the sample in the Midwest are 18 to 35 years old, 40.9\% in the
Northeast are 18 to 35, 25\% in the South are 18 to 35, 36.4\% in the
West are 18 to 35.''

The number 2 in the prop.table function is what directs R to compute
column percents. In matrix notation, the rows get mentioned first and
the columns get mentioned second, so a 2 indicates that we want column
percents.}

\end{frame}

\begin{frame}{Odds Ratios}

Let's go back to the text book for an example of odds ratios. Example
10.16 on pp.~533-535 of the Ott textbook uses the following data.

\begin{figure}[htbp]
\centering
\includegraphics{./figures/Table_10-19.jpg}
\caption{}
\end{figure}

\note{}

\end{frame}

\begin{frame}[fragile]{Odds Ratios}

The R code for entering the data in Example 10.16 on pp.~533-535 of the
Ott textbook is

\begin{Shaded}
\begin{Highlighting}[]
\NormalTok{counts=}\KeywordTok{matrix}\NormalTok{(}\KeywordTok{c}\NormalTok{(}\DecValTok{250}\NormalTok{,}\DecValTok{400}\NormalTok{,}\DecValTok{750}\NormalTok{,}\DecValTok{1600}\NormalTok{),}\DataTypeTok{nrow=}\DecValTok{2}\NormalTok{)}
\KeywordTok{rownames}\NormalTok{(counts) <-}\StringTok{ }\KeywordTok{c}\NormalTok{(}\StringTok{"Low"}\NormalTok{,}\StringTok{"High"}\NormalTok{)}
\KeywordTok{colnames}\NormalTok{(counts) <-}\StringTok{ }\KeywordTok{c}\NormalTok{(}\StringTok{"Favorable"}\NormalTok{,}\StringTok{"Unfavorable"}\NormalTok{)}
\NormalTok{counts}
\end{Highlighting}
\end{Shaded}

\begin{verbatim}
##      Favorable Unfavorable
## Low        250         750
## High       400        1600
\end{verbatim}

\note{}

\end{frame}

\begin{frame}{R function for Odds Ratio and Relative Risk (package:
mosaic)}

oddsRatio(counts,verbose=TRUE)

\note{Running the oddsRatio function will require you to install the
package called mosaic first, but it does a nice job of computing the
proportions, relative risk, odds, and odd ratio as well as the
confidence intervals for the relative risk and odds ratio.

bottom panel note: For Example 10.16 on pp.~533-535 of the Ott textbook}

\end{frame}

\begin{frame}[fragile]{R output for oddsRatio(counts,verbose=TRUE)}

\begin{Shaded}
\begin{Highlighting}[]
\NormalTok{Proportions}
       \NormalTok{Prop. }\DecValTok{1}\NormalTok{:}\StringTok{  }\FloatTok{0.25} 
       \NormalTok{Prop. }\DecValTok{2}\NormalTok{:}\StringTok{  }\FloatTok{0.2} 
     \NormalTok{Rel. Risk:}\StringTok{  }\FloatTok{0.8} 

\NormalTok{Odds}
        \NormalTok{Odds }\DecValTok{1}\NormalTok{:}\StringTok{  }\FloatTok{0.3333} 
        \NormalTok{Odds }\DecValTok{2}\NormalTok{:}\StringTok{  }\FloatTok{0.25} 
    \NormalTok{Odds Ratio:}\StringTok{  }\FloatTok{0.75} 

\DecValTok{95} \NormalTok{percent confidence interval:}
\StringTok{     }\FloatTok{0.6965} \NormalTok{<}\StringTok{ }\NormalTok{RR <}\StringTok{ }\FloatTok{0.9189} 
     \FloatTok{0.6263} \NormalTok{<}\StringTok{ }\NormalTok{OR <}\StringTok{ }\FloatTok{0.8981} 
\end{Highlighting}
\end{Shaded}

\note{With the option verbose=TRUE, we get all the output we want here.
We get the proportions of a favorable response for both the low and high
stress jobs along with their ratio, the relative risk with row 2 in the
numerator; .2 divided by .25 equals .8.

We get the odds of a favorable response for the low stress job as 250
divided by 750, which is 0.3333, and the odds of a favorable response
for the high stress job, which is 400 divided by 1600, which is 0.25.

And, of course, we get the ratio of those odds, with the odds for row 2
in the numerator as .25 divided by .3333 to get .75.

95\% percent confidence intervals for the relative risk and odds ratio
are also displayed. The level of confidence can be adjusted in the same
manner as other functions in R.

The oddsRatio function processes the 2x2 table with the expectation that
the successes are located in column 1 and the treatment of interest is
on row 2. The treatment of interest is the one that occupies the
numerator in the odds ratio. It's the one that leads in the
interpretation.

This is why the odds ratio you see here, .75, does NOT match the one
computed in the textbook, 1.333. Well, actually it does match, its just
that the comparison is from the perspective of Low Job Stress rather
than the High Job Stress. 0.75 (3/4) is the recoprical of 1.333 (4/3).

bottom panel note: Example 10.16 on pp.~533-535 of the Ott textbook}

\end{frame}

\begin{frame}{Odds Ratio: Interpretation Options when the OR = 0.75}

\begin{enumerate}
\def\labelenumi{\arabic{enumi}.}
\tightlist
\item
  As a multiple
\end{enumerate}

``The odds of a favorable response for employees in a high stress job
are 0.75 times as large as the odds of a favorable response for
employees in a low stress job.''

or

``The odds of a favorable response for employees in a high stress job
are only three-fourths of the odds for employees in a low stress job.''

\note{Odds ratios can be interpreted in a variety of ways. In any case,
one must proceed with caution when interpreting odds ratios, because
they can so easily be misrepresented or misunderstood. Take some time to
read these interpretations carefully.

bottom panel note: Example 10.16 on pp.~533-535 of the Ott textbook}

\end{frame}

\begin{frame}{Odds Ratio: Interpretation Options when the OR = 0.75}

\begin{enumerate}
\def\labelenumi{\arabic{enumi}.}
\setcounter{enumi}{1}
\tightlist
\item
  As a percent
\end{enumerate}

``The odds of a favorable response for employees in a high stress job
are only 75\% of the odds for employees in a low stress job.''

or

``The odds of a favorable response for employees in a high stress job
are 25\% less than the odds of a favorable response for employees in a
low stress job.''

\note{I like the second option here and I believe it is more common to
express an odds ratio as a percent when its less than 1.

bottom panel note: Example 10.16 on pp.~533-535 of the Ott textbook}

\end{frame}

\begin{frame}{Interpreting the OR Confidence Interval}

Recall output from R

95 percent confidence interval: 0.6263 \textless{} OR \textless{} 0.8981

``With 95\% confidence, the odds of a favorable response from an
employee in a high stress job are 63 to 90 percent as high as for an
employee in a low stress job.''

\note{An odds ratio of 1 would tell us that the odds of an event for the
first group are identical to the odds for the second group. When we see
a confidence interval that does not contain 1, we can conclude that
there is a statistically significant relationship between the two
categorical factors.

We could have equally said ``With 95\% confidence, that the odds of a
favorable response from an employee in a high stress job are 10 to 37
percent less than for an employee in a low stress job.''

bottom panel note: Example 10.16 on pp.~533-535 of the Ott textbook}

\end{frame}

\begin{frame}[fragile]{Let's reconstruct the 2x2 table so our output
matches the textbook example output}

\begin{Shaded}
\begin{Highlighting}[]
\NormalTok{counts=}\KeywordTok{matrix}\NormalTok{(}\KeywordTok{c}\NormalTok{(}\DecValTok{400}\NormalTok{,}\DecValTok{250}\NormalTok{,}\DecValTok{1600}\NormalTok{,}\DecValTok{750}\NormalTok{),}\DataTypeTok{nrow=}\DecValTok{2}\NormalTok{)}
\KeywordTok{rownames}\NormalTok{(counts) <-}\StringTok{ }\KeywordTok{c}\NormalTok{(}\StringTok{"High"}\NormalTok{,}\StringTok{"Low"}\NormalTok{)}
\KeywordTok{colnames}\NormalTok{(counts) <-}\StringTok{ }\KeywordTok{c}\NormalTok{(}\StringTok{"Favorable"}\NormalTok{,}\StringTok{"Unfavorable"}\NormalTok{)}
\NormalTok{counts}
\end{Highlighting}
\end{Shaded}

\begin{verbatim}
##      Favorable Unfavorable
## High       400        1600
## Low        250         750
\end{verbatim}

\note{By entering the 2x2 table into R such that the frequencies for the
Low Stress Job are in row 2, so that R puts them in the numerator of the
odds ratio, we can replicate the output for example 10.16 in the
textbook.

bottom panel note: Example 10.16 on pp.~533-535 of the Ott textbook}

\end{frame}

\begin{frame}[fragile]{R output for Example 10.16 (again)}

\begin{Shaded}
\begin{Highlighting}[]
\NormalTok{Proportions}
       \NormalTok{Prop. }\DecValTok{1}\NormalTok{:}\StringTok{  }\FloatTok{0.2} 
       \NormalTok{Prop. }\DecValTok{2}\NormalTok{:}\StringTok{  }\FloatTok{0.25} 
     \NormalTok{Rel. Risk:}\StringTok{  }\FloatTok{1.25} 

\NormalTok{Odds}
        \NormalTok{Odds }\DecValTok{1}\NormalTok{:}\StringTok{  }\FloatTok{0.25} 
        \NormalTok{Odds }\DecValTok{2}\NormalTok{:}\StringTok{  }\FloatTok{0.3333} 
    \NormalTok{Odds Ratio:}\StringTok{  }\FloatTok{1.333} 

\DecValTok{95} \NormalTok{percent confidence interval:}
\StringTok{     }\FloatTok{1.088} \NormalTok{<}\StringTok{ }\NormalTok{RR <}\StringTok{ }\FloatTok{1.436} 
     \FloatTok{1.113} \NormalTok{<}\StringTok{ }\NormalTok{OR <}\StringTok{ }\FloatTok{1.597}
\end{Highlighting}
\end{Shaded}

\note{Notice now that the odds ratio and confidence interval bounds for
the odds ratio now match the values given on page 534 of Ott's textbook.

bottom panel note: Example 10.16 on pp.~533-535 of the Ott textbook}

\end{frame}

\begin{frame}{Odds Ratio: Interpretation Options when the OR = 1.333}

\begin{enumerate}
\def\labelenumi{\arabic{enumi}.}
\tightlist
\item
  As a multiple
\end{enumerate}

``The odds of a favorable response for employees in a low stress job are
1.33 times the odds of a favorable response for employees in a high
stress job.''

\note{bottom panel note: Example 10.16 on pp.~533-535 of the Ott
textbook}

\end{frame}

\begin{frame}{Odds Ratio: Interpretation Options when the OR = 1.333}

\begin{enumerate}
\def\labelenumi{\arabic{enumi}.}
\setcounter{enumi}{1}
\tightlist
\item
  As a percent
\end{enumerate}

``The odds of a favorable response for employees in a low stress job are
only 133\% of the odds for employees in a high stress job.''

or

``The odds of a favorable response for employees in a low stress job are
33\% more than the odds of a favorable response for employees in a high
stress job.''

\note{bottom panel note: Example 10.16 on pp.~533-535 of the Ott
textbook}

\end{frame}

\begin{frame}{Interpreting the OR Confidence Interval}

Recall output from R

95 percent confidence interval: 1.113 \textless{} OR \textless{} 1.597

``With 95\% confidence, the odds of a favorable response from an
employee in a low stress job are 11 to 60 percent higher than for an
employee in a high stress job.''

\note{bottom panel note: Example 10.16 on pp.~533-535 of the Ott
textbook}

\end{frame}

\begin{frame}

\end{frame}


\providecommand{\tightlist}{%
  \setlength{\itemsep}{0pt}\setlength{\parskip}{0pt}}


\end{document}
